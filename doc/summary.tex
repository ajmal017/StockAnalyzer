\documentclass[ %
paper=a4,					% Papierformat
11pt,						% Schriftgröße
DIV15,						% Teilung (DIV9 für 9er-Teilung) für Satzspiegel
BCOR=5mm,					% Bindekorrektur in mm
twocolumn=false,		% onecolumn = einspaltig, twocolumn = zweispaltig
%openright,				% erste Seite des Kapitels immer rechts			
titlepage, 				% es wird eine Titelseite verwendet
%parindent,				% neuer Absatz wird links eingerückt (Standard)
parskip=half-,				% neuer Absatz nach Leerzeile und nicht eingerückt
twoside=false,			% twoside für zweiseitige Dokumente
headings=normal, 		% Größe der Überschriften verkleinern
headinclude,				% Lage von Kopfzeile in Satzspiegel mit einberechnen
footexclude,				% Lage von Fußzeile in Satzspiegel nicht mit einberechnen
captions=tableheading,	% Für einen größeren Abstand zwischen Überschrift und Tabelle/Abbildung
numbers=noenddot,
%subsectionprefix,		% 
%fleqn,					% für linksbündige statt zentrierte mathematische Gleichungen
%leqno,					% für Gleichungsnummern links statt rechts von jeder nummerierten Gleichung
%toc=listofnumbered,
toc=bib,
%listof=totoc, 			% Verzeichnisse im Inhaltsverzeichnis aufführen
%bibliography=totocnumbered, 	% Literaturverzeichnis im Inhaltsverzeichnis aufführen totoc, totocnumbered
%showframe,
%index=totoc,			% Index im Inhaltsverzeichnis aufführen
%draft						% Status des Dokuments (final/draft)
]{scrreprt}

\usepackage[utf8]{inputenc}
\usepackage[ngerman]{babel}
\usepackage[T1]{fontenc} 

\usepackage{amsmath, amsthm, amssymb}		% Mathematik-Umgeungen und Symbole
\usepackage[]{siunitx}
\usepackage[]{xcolor}

\typearea[current]{calc}	            		% Satzspiegel neu berechnen
\usepackage[top=3.2cm,bottom=3cm,left=3cm,right=2.5cm]{geometry}	

\usepackage[automark,pagestyleset=KOMA-Script,headsepline]{scrlayer-scrpage}
\ofoot*[\normalfont\thepage]{\normalfont\thepage}
\cfoot*[]{}
%\automark[section]{chapter}

\usepackage{lmodern}

\newcommand{\englishName}[1]{\textit{#1}}
\newcommand{\caution}[1]{\begin{center} \textbf{#1} \end{center}}
\newcommand{\formulaText}[1]{\text{#1}}

\newcommand{\earnings}[0]{Einnahmen (\englishName{earnings}) }
\newcommand{\sales}[0]{Umsätze (\englishName{sales}) }
\newcommand{\netMargin}[0]{Nettogewinn (\englishName{net margin}) }
\newcommand{\assetTurnover}[0]{Vermögensumsatz (\englishName{asset turnover}) }
\newcommand{\financialLeverage}[0]{finanzielle Hebelwirkung (\englishName{financial leverage}) }
\newcommand{\shareholdersEquity}[0]{Eigenkapital (\englishName{shareholder's equity}) }
\newcommand{\assets}[0]{Vermögen (\englishName{assets}) }
\newcommand{\returnOnEquity}[0]{Eigenkapitalrenidte (\englishName{return on equity}) }
\newcommand{\returnOnAssets}[0]{Kapitalrenidte (\englishName{return on assets}) }
\newcommand{\freeCashFlow}[0]{freie Cashflow (\englishName{free cash flow}) }
\newcommand{\inventories}[0]{Inventar (\englishName{inventories}) }

\newenvironment{formel}{\sffamily\begin{equation}}{\end{equation}}

\newcommand{\formulaNote}[1]{
    \begin{center}
        \vspace*{-1em}
        \colorbox{lightgray}{
            \begin{minipage}[t]{0.9\textwidth}
                \sffamily
                \begin{equation*}
                    #1
                \end{equation*}
            \end{minipage}
        }
    \end{center}
}


%\newenvironment{note}{\begin{center}\colorbox{red}{\begin{minipage}[t]{0.9\textwidth}#1\end{minipage}}}{\end{center}}

\title{Handbuch Unternehmensanalyse}
\author{Mathias Ferstl}

\pdfinfo{
   /Author (\@author)
   /Title  (\@title)
   /Subject (Unternehmensanalyse)
   /Keywords (Aktien;Unternehmen;Analyse)
}

\begin{document}

\maketitle

\tableofcontents


%
%%
%%%
%%%%
\chapter{Unternehmensbilanzen lesen}

Die börsennotierten Unternehmen berichten ihre Ergebnisse überlicherweise jedes Quartal  (\englishName{10-Q}) und legen am Ende eines Geschäftsjahres auch eine Jahresbilanz (\englishName{10-K}) vor.
Die wichtigsten Informationen finden sich dabei in drei verschiedenen Blättern, die nachfolgend näher betrachtet werden.

%
%%
%%%
\section{Balance Sheet}
Das \englishName{Balance Sheet} gibt einen Überblick über das Vermögen und die Schulden eines Unternehmens und beschreibt somit den aktuellen finanziellen Zustand der Firma.
Wie der Name andeutet, befinden sich die Zahlen in diesem Blatt im Gleichgewicht (\textit{balance}), da folgendes Gleichgewicht gilt:
\formulaNote{\formulaText{Assets} - \formulaText{Liablilities} = \formulaText{Equity}}


%
%%
\subsection{Asset Accounts}

\begin{description}
    \item[Current Assets] sind die Vermögenswerte, die innerhalb eines Finanzzykluses (normalerweise 1 Jahr) zu Geld gemacht werden.
        Diese umfassen \englishName{cash and equivalents}, \englishName{short-term investments}, \englishName{accounts receivable} und \englishName{inventories}.
        \begin{description}
            \item[Cash and Equivalents and short-term-Investments] beschreibt, was das Unternehmen kurzfristig an Geld verfügbar hat.
            \item[Accounts Receivable] umfasst das Geld, das das Unternehmen noch nicht erhalten hat, aber erwartet demnächst zu erhalten.
                Wenn ein Produkt bereits an den Kunen ausgeliefert wurde, dessen Zahlung aber noch nicht eingegangen ist, dann wird der in der Kasse noch fehlende Betrag hier addiert.
                Wenn dieser Posten deutlich schneller wächst als der Umsatz, dann verbucht das Unternehmen viele Einnahmen, obwohl es dafür noch kein Geld erhalten hat.
                Dies kann zu Problemen führen, wenn das Unternehmen einen Teil der erwarteten Einnahmen gar nicht erhält.
                Manchmal findet sich im \englishName{Balance Sheet} auch ein Posten mit der Bezeichnung \englishName{allowance for doubtful accounts}, der den Betrag beschreibt, der dem Unternehmen zwar geschuldet wird, den das Unternehmen aber nicht mehr erwartet vom Kunden tatsächlich zu erhalten.
            \item[Inventories] umfasst Kapital, das in Form von Materialien, teilweise fertigen Produkten, hergestellten aber noch nicht verkauften Produkten etc. vorhanden ist.
                Der angegebene Betrag muss immer mit Vorsicht betrachtet werden, da die hierzu gezählten Produkte und Materialien zwar den angegebenen Preis haben können, es aber nicht sicher ist, dass bei einem plötzlichen Verkauf dieser Produkte und Materialien auch dieser Preis erzielt werden kann.
                Wenn ein Unternehmen weniger Material und fertige Produkte \glqq herumliegen \grqq\ hat, dann könnte es profitabler sein, da es weniger Kapital in diesen Vermögenswerten gebunden hat.
                Eine Aussage hierzu liefert der Kennwert \textbf{\englishName{inventory turnover}}, der angibt, wie oft das gesamte Inventar während eines Finanzzykluses umgeschlagen wurde.
                \begin{formel}
                    \formulaText{inventory turnover} = \frac{\formulaText{cost of goods sold}}{\formulaText{inventories}}
                \end{formel}
        \end{description}
    \item[Asset Accounts] sind langfristige Vermögenswerte, die nicht innerhalb eines Finanzzykluses zu Geld gemacht werden.
        Hierzu zählt Folgendes:
        \begin{description}
            \item[PP\&E Property, Plant and Equipment] umfasst das Kapital, das in Fabriken, Gebäuden, Grundstücken, Maschinen etc. gebunden ist.
                Wenn man diese Zahl mit dem Wert der gesamten Vermögenswerte (\englishName{Total Assets}) vergleicht, dann kann man ein Gefühl bekommen, wie kapitalintensiv das Unternehmen ist. 
            \item[Investments] ist das Kapital, das in langfristigen Anleihen und Aktien anderer Unternehmen steckt.
                Wenn dies einen signifikaten Anteil des Gesamtvermögens eines Unternehmens ausmacht, dann sollte man heraufinden, wo dieses Geld steckt, wie sicher es ist und wie lange es gebunden ist. 
            \item[Intangible Assets] beschreibt alle immateriellen Güter. Hierzu zählt z.B. auch der sog. \englishName{Goodwill}. 
                Dieser beschreibt das Geld, das bei einer Übernahme des Unternehmens notwendig ist und den Sachwert der Firma übersteigt, da z.B. auch der Ruf und die Marke \glqq gekauft\grqq\ werden.   
                Generell sollte dieser Wert mit Skepsis betrachtet werden, da Firmen den Wert generell sehr hoch ansetzen und es nicht sicher ist, dass dieser Betrag auch gerechtfertigt ist.
        \end{description}
\end{description}

%
%%
\subsection{Liabilitiy Accounts}

Dieser Anteil umfasst alle Verflichtungen (Schulden) des Unternehmens.
Diese werden in zwei Rubriken unterteilt.

\begin{description}
    \item[Current Liabilities] Geld, das das Unternehmen innerhalb eines Finanzzykluses ausgeben wird. 
        Dies umfasst folgende Posten.
        \begin{description}
            \item[Accounts Payable] Geld, das das Unternehmen Anderen schuldet, z.B. in Form offener Rechnungen. 
                Wenn das Unternehmen viele Rechnungen erst später bezahlt und das Geld deshalb noch in der Firma bleibt, dann hat dies positive Auswirkungen auf den \englishName{cash flow}. 
            \item[Short-Term-Borrowings] kurzfristig geliehenes Geld, das innerhalb eines Jahres zurückgezahlt werden muss. 
                Wenn das Unternehmen viel Geld kurzfristig geliehen hat, aber nicht so viel Geld kurzfristig verfügbar hat, dann kann das Unternehmen kurzfristige Geldprobleme haben.
                siehe auch \englishName{current ratio} in Kapitel \ref{sec:current_ratio}.
        \end{description}
    \item[Noncurrent Liabilities] Geld, das das Unternehmen nicht innerhalb des eines Finanzzykluses ausgibt und deshalb langfristige Verbindlichkeiten beschreiben.
        \begin{description}
            \item[Long-Term-Debt] Geld, das das Unternehmen z.B. der Bank schuldet, aber nicht in nächster Zeit zurückzahlen muss. 
        \end{description}
\end{description}

%
%%
\subsection{Stockholder's equity}

Dieser Posten umfasst das Eigenkapital des Unternehmens, das in Form von Aktien im Umlauf ist.
Die meisten der aufgeführten Unterpunkte haben eine geringe Relevanz.
Der einzige Eintrag, der hier einen Blick wert ist, ist unter dem Namen \englishName{Retained Earnings} aufgeführt.
\begin{description}
    \item[Retained Earnings] Kapital, das das Unternehmen über seine gesamte Lebensdauer generiert hat, abzüglich Dividendenzahlungen und Aktienrückkäufen, da dieses Geld zurück an die Aktionäre gewandert ist.
        Jedes Jahr, in dem das Unternehmen Gewinn macht und diesen nich vollständig an die Aktionäre ausschüttet, steigt dieser Betrag.
        Wenn das Unternehmen bisher kein Geld behalten sondern sogar verloren bzw. mehr ausgegeben hat, dann hat dieser Eintrag einen negativen Wert.
        Generell kann hier geprüft werden, wie das Unternehmen Geld über die Jahre einnimmt.
\end{description}

%
%%
%%%
\section{Income Statement}
Dieses Blatt beinhaltet die Einnahmen und Ausgaben innerhalb eines Finanzzykluses und beschreibt somit die aktuelle Entwicklung des Unternehmens.

\begin{description}
    \item[Revenue/Sales] Umsatz der Firma innerhalb eines Finanzzykluses.
        Große Unternehmen brechen den Umsatz auch oft auf einzelne Produkte, Geschäftsbereiche oder Regionen auf der Erde herunter.
        Firmen können ihren Umsatz aber zu verschiedenen Zeitpunkten verbuchen, da z.B. ein Softwarehersteller hier einen großen Geldbetrag vermerken kann, wenn ein Produkt an den Kunden ausgeliefert wird, wohingegen eine Service-Firma den Umsatz über die gesamte Vertragslaufzeit gleichmäßig verteilt.
    \item[Cost of Sales] Geld, das aufgewendet werden musste um den aufgeführten Umsatz zu generieren.
        Manchaml aus unter dem Namen \englishName{Cost of goods sold} aufgeführt. 
    \item[Gross Profit] enspricht der Differenz aus Umsatz und den Aufwendungen für den Umsatz und beschreibt das Geld aus dem gesamten Umsatz, das nicht für die Bereit- und Herstellung der Produkte erforderich ist.
        \begin{formel}
            \formulaText{Gross Profit} = \formulaText{Revenue} - \formulaText{Cost of Sales}
        \end{formel}
        Der prozentuale Anteil wird auch als \englishName{Gross margin} bezeichnet. 
        \formulaNote{\formulaText{Gross margin} = \frac{\formulaText{Gross Profit}}{\formulaText{Revenue}}} 
    \item[Selling, General and Administrative Expenses (SG\&A)] Ausgaben für Marketing, Verwaltung etc.
        Bei einer Firma mit großen Vertriebsnetzwerk, das viele Beschäftigte im Vertrieb hat ist dieser Posten oftmals groß, wobei diese Unternehmen meist aufgrund der höheren erzielten Produktepreise auch eine hohe \englishName{gross margin} haben.
    \item[Depreciation and Amortization] Abschreibungen von Gebäuden etc.
    \item[Nonrecurring Charges/Gains] Einmalzahlungen und -ausgaben.
        Dieser Posten sollte in der Regel einen geringen Anteil im Vergleich zum Umsatz darstellen.
        Falls ein Unternehmen im \englishName{Income Statement} immer wieder nicht-unwesentiche Beträge als Einmalzahlungen aufführt, dann sollte man hellhörig werden!
        Es könnte sein, dass die Firma dubiose Ausgaben vermehrt als Einmalzahlungen deklariert.
    \item[Operating Income] Unternehmensergebnis. Dieser Wert wird auch genutzt, um die \englishName{operation margin} zu ermitteln, die eine Grundlage für den Vergleich verschiedener Firmen und Industrien bildet.
        \begin{formel}
            \formulaText{Operating Income} = \formulaText{Revenues} - \left( \formulaText{Cost of Sales} + \formulaText{Cost of Sales}\right)
        \end{formel}
    \item[Interest Income/Expense] Ausgaben für bzw. Einnahmen aus Anleihen. 
        Manchmal werden die Einnahmen und Ausgaben auch unter dem Punkt \englishName{net interest income} verrechnet.
    \item[Taxes] Steuern.
        Wenn dieser Wert über die Jahre stark schwankt, dann versucht das Unternehmen möglicherweise mit Steuertricks Geld zu sparen. 
        Es ist deshalb möglich, dass die Firma das \glqq gesparte\grqq\ Geld irgendwann noch nachzahlen muss.
    \item[Net Income] Gewinn nach Abzug aller Ausgaben.
        Dieser Wert wird meistens von den Unternehmen hervorgehoben, jedoch sollte man ihn mit Vorsicht genießen, da er durch Einmahlzahlungen und Kapitalerträge verzerrt werden kann.
    \item[Number of Shares] Anzahl der im Umlauf befindlichen Aktien, die zur Berechnung des Gewinns pro Aktie genutzt werden.
        Hier gibt es verschiedene Arten
        \begin{description}
            \item[basic] beschreibt die Anzahl der tatsächlich im Umlauf befindlichen Aktien. Dieser Wert kann aber ignoriert werden und stattdessen der nachfolgende betrachtet werden.
            \item[diluted] verwässterte Anzahl an Aktien, die auch Optionen, die in Aktien umgewandelt werden können, wandelbare Anleihen (\englishName{convertible bonds}) und Ähnliches berücksichtigt.
                Dieser Wert gibt an, auf welchen Anteil am gesamten Unternehmen der eigene Anteil in Form erworbener Aktien sinken kann. 
        \end{description}
    \item[Earnings per Share] (EPS) Gewinn pro Akie. Diese Kennzahl wird auch oft für den Entwicklung über die letzten Jahre genutzt. 
        Analysten schätzen meist auch das EPS der nächsten Quartale und des kommenden Jahres.  
        \formulaNote{\formulaText{EPS} = \frac{\formulaText{Net Income}}{\formulaText{Number of Shares}}}    
\end{description}


%
%%
%%%
\section{Statement of Cash Flows}
Dieses Blatt zeigt auf, wie viel Geld tatsächlich in und aus dem Unternehmen geflossen ist.
Hier taucht nur Geld auf, das innerhalb der Finanzperiode tatsächlich in das oder aus dem Unternehmen ging.
So kann z.B. Geld, das ein Kunde erst in der Zukunft bezahlen wird, im Income Statement bereits enthalten sein, im Statement of Cash Flows ist es jedoch nicht eingerechnet, da es das Unternehmen noch nicht erhalten hat.

Einer der wichtigsten Kennwerte, der \freeCashFlow lässt sich aus den hier aufgeführten Zahlen ermitteln.
\formulaNote{\formulaText{Free cash flow} = \formulaText{operating cash flow} - \formulaText{Capital Expenditures}}

%
%%
\subsection{Cash Flow from operating activities}

Dieser Anteil beschreibt das Geld, das das Unternehmen aus seinem Kerngeschäft, z.B. Verkauf von Produkten, tatsächlich eingenommen hat.
Die üblichen Einträge in dieser Auflistung sind nachfolgend beschrieben.
\begin{description}
    \item[Net Income] Nettogewinn. Diese Zahl ist aus dem \englishName{Income Statement} übernommen.
    \item[Depreciation and Amortization] Abschreibungen. Ist ebenfalls aus dem \englishName{Income Statement} übernommen.
    \item[Tax Benefits from Employee Stock Plans] Aktienoptionen für Angestellte des Unternehmens
    \item[Changes in working captal] Veränderung in \englishName{Accounts Receivable}, \englishName{Accounts Payable} und \englishName{Inventories}
    \item[One-Time charges] Ebenfalls aus dem \englishName{Income Statement} übernommen (unter \englishName{Nonrecurring Charges}) 
\end{description}

Es können auch noch mehr oder feiner untergliederte Einträge vorhanden sein.
Die Summe aller aufgelisteten Beträge beschreibt das \textbf{\englishName{Net cash provided by operating activities}} und wird oft auch als \englishName{operating cash flow} bezeichnet.

%
%%
\subsection{Cash Flow from investing activities}

Dieser Anteil beschreibt die Aufwendungen des Unternehmens für Investitionen und Ähnliches.
\begin{description}
    \item[Capital Expenditures] Alle notwendigen Ausgaben um das Geschäft am Laufen zu halten. 
        Hierzu zählen z.B. die Ausgaben für \englishName{PP\&E}.
    \item[Investment Proceeds] Anlagen des Unternehmens  
\end{description}

%
%%
\subsection{Cash Flow from financing activities}

Dieser Abschnitt beinhaltet alle Transaktionen mit den Unternehmenseigner und Kredtigebern.
\begin{description}
    \item[Dividends Paid] Dividendenzahlungen an die Aktionäre
    \item[Issuance/Purchase of Common Stock] Eingenommenes Geld aus der Ausgabe neuer Aktien oder aufgewendetes Geld für Aktienrückkäufe.
    \item[Issuance/Repayments of Debt] Aufnahme von Krediten und Rückzahlungen  
\end{description}



\chapter{Unternehmensanalyse in 5 Schritten}
\begin{itemize}
    \item Wachstum
    \item Profitabilität
    \item Finanzielle Gesundheit
    \item Risiken
    \item Management
\end{itemize}

\section{Wachstum}

Ein Unternehmen hat grundsätzlich mehrere Möglickeinten für zukünftiges Wachstum
\begin{itemize}
    \item mehr Produkte verkaufen
    \item höhere Preise für die Produkte festlegen
    \item neue Produkte und Dienstleistungen anbieten
    \item andere Unternehmen übernehmen (kein nachhaltiges Wachstum!)
\end{itemize}

Für das zukünftige Wachstum sollte man sich klar machen, welche dieser Möglichkeiten das Unternehmen für sein zukünftiges Wachstum nutzen kann.

In den Kennzahlen des Unternehmens sind die \earnings hierfür nicht immer aussagekräftig. 
Stattdessen sollten die Umsätze \sales analysiert werden, um einen Eindruck vom bisherigen Wachtum zu erhalten.
\caution{Bisheriges Wachstum gibt keine Aussage über zukünftiges Wachstum}
Wenn das Unternehmen bisher sehr stark gewachsen ist, dann wird sich dieses Wachstum üblicherweise nicht in den nächsten Jahren so rasant fortsetzen.

Wenn sich in den Unternehmenszahlen zeigt, dass das Wachstum der \earnings über einen längeren Zeitraum das Wachstum der \sales übersteigt, dann könnte in den Unternehmenszahlen etwas faul sein!
Als Grundlage für die Analyse können das Verhältnis von Umsatz zu Einnahmen
\begin{equation}
    \frac{\formulaText{sales}}{\formulaText{revenues}}
\end{equation}
sowie die Entwicklung des \englishName{cashflow from operations} herangezogen werden.

\section{Profitabilität}
Die Profitabilität eines Unternehmens gibt Aufschluss darüber, wie viel Geld eine Firma mit dem investierten Kapital umsetzt.
Hierfür kann wieder das Wachstum des \englishName{cashflow from operations} betrachtet werden.

Darüber hinaus geben die Kennwerte zum \englishName{return on capital} Aufschluss

%
%%
\subsection{RoA -- \englishName{Return on Assets}}

Diese Kennzahl gibt an, wie hoch die erzielten Gewinne das Unternehmen pro Dollar ihres Vermögens umsetzen kann.
\begin{equation}
    \formulaText{RoA} = \underbrace{\frac{\formulaText{net income}}{\formulaText{sales}}}_{\formulaText{net margin}} \cdot \underbrace{\frac{\formulaText{sales}}{\formulaText{assets}}}_{\formulaText{asset turnover}} = \frac{\formulaText{net income}}{\formulaText{assets}}
\end{equation}
Je höher dieser Kennwert ist, desto besser ist das Unternehmen.
Der \netMargin gibt an, wie viel das Unternehmen nach Abzug aller Kosten für ihr Geschäft behält.
Die zweite Komponente \assetTurnover gibt Auskunft, wie effizient eine Firma bei der erwirtschaftung des Umsatzes pro Dollar des Vermögens ist.

%
%%
\subsection{RoE -- \englishName{Return on Equity}}

\begin{equation}
    \formulaText{RoE} = \formulaText{RoA} \cdot \underbrace{\frac{\formulaText{assets}}{\formulaText{shareholder's eqity}}}_{\formulaText{financial leverage}} = \frac{\formulaText{net income}}{\formulaText{shareholder's equity}}
\end{equation}
Der Wert für den \financialLeverage kann nicht pauschal beurteilt werden, da übliche Werte von Branche zu Branche unterschiedlich sind. 
Um den Wert beurteilen zu können, sollte er daher immer im Vergleich zu den Werten von anderen Unternehmen aus der gleichen Branche betrachtet werden.
Generell ist ein nieriger Wert besser, da das Unternehmen dann mehr \shareholdersEquity relativ zum \assets besitzt.
Folgende Anhaltswerte können genutzt werden
\begin{description}
    \item[Nicht-Finanzbranche] Wenn die \returnOnEquity in den letzten Jahren konstant über $\SI{10}{\percent}$ ist und das Unternehmen gleichzeitig keinen extremen \financialLeverage ausweist, dann ist das ein \textbf{gutes Zeichen}
    \item[Finanzbranche] Hier ist der \financialLeverage immer sehr groß, weshalb auf eine \returnOnEquity der letzten Jahre von mindestens $\SI{12}{\percent}$ als Ziel angesetzt werden sollte.
\end{description}
\caution{\returnOnEquity über $\SI{40}{\percent}$? $\rightarrow zu schön um wahr zu sein$}
Unternehmen mit einer extrem hohen \returnOnEquity wurden oftmals vor kurzer Zeit von anderen Unternehmen abgespalten oder haben sehr viele Aktien zurückgekauft. 
Es ist auch möglich, dass die Finanzstruktur des Unternehmens diesen Wert verzerrt.
Deshalb bedeutet ein extrem hoher Wert nicht automatisch, dass das Unternehmen deutlich besser als die ganze Branche wirtschaftet. 

%
%%
\subsection{RoIC -- \englishName{Return on invested Capital}}

Diese Kennzahl ist schwer zu berechnen, weshabl hier auf eine Beschreibug verzichtet wird.
\begin{equation}
    \formulaText{RoIC} = \frac{\formulaText{net operating profit after taxes (NOPAT)}}{\formulaText{invested capital}}
\end{equation}
mit
\begin{align}
    \begin{split}
        \formulaText{invested capital} =\ &\formulaText{total assets}\\ &- \formulaText{accounts payable \& other current assets} \\ &- \formulaText{excess cash (cash needed for day-to-day business)}\\ &- \formulaText{Goodwill}
    \end{split}
\end{align}


Eine hohe Aussagekraft darüber, wie das Unternehmens wirtschaftet, gibt der \textbf{\freeCashFlow}.
Er gibt an, wie viel Geld das Unternehmen nach Abzug aller Ausgaben am Ende noch übrig hat und gibt somit an, wie viel Geld dem Unternehmen entnommen werden könnte, ohne das Geschäft zu beeinträchtigen.
Der Wert berechnet sich nach
\begin{align}
    \formulaText{free cash flow} = \formulaText{cash flow from operations} - \formulaText{capital spending}
\end{align}
Um das Unternehmen zu beurteilen, kann das Verhältnis des Werts zum Umsatz betrachtet werden.
\formulaNote{\frac{\formulaText{free cash flow}}{\formulaText{sales}} > \SI{5}{\percent} \rightarrow \formulaText{solides Unternehmen}}

%
%%
%%%
\section{Finanzielle Gesundheit}

Bei der Beurteilung der finanziellen Gesundheit eines Unternehmens werden die gesamten Vermögenswerte betrachtet, die im Balance Sheet aufgeführt sind.
Das Verhältnis von Schulden oder Vermögenswerten zum Eigenkapital gibt Auskunft, wie fest das Unternehmen im Sattel sitzt und eine ggf. kommende Krise meistern kann.

%
%%
\subsection{Financial leverage}

Die Kennzahl beschreibt das Verhältnis der Vermögenswerte zum Eigenkapital. 
Für die meisten Branchen sollte dieses Verhältnis nicht über 4 liegen, da das Unternehmen im Falle einer Krise wahrscheinlich seine Schulden nicht weiter tilgen kann und zusätzliche Kreidte aufnehmen muss oder in die Insolvenz abrutschen kann.
\formulaNote{\frac{\formulaText{assets}}{\formulaText{eqity}} > 4 \rightarrow \formulaText{hohes Risiko}}

%
%%
\subsection{Times interest earned}

Diese Kennzahl beschreibt, wie oft das Unternehmen mit seinen Gewinnen die Schulden hätte zurückzahlen können. 
Ein hoher Wert deutet dabei auf ein geringes Risiko hin. 
Der Wert muss aber immer Branchenabhängig betrachtet werden.
\begin{equation}
    \formulaText{Times interest earned} = \frac{\formulaText{EBIT}}{\formulaText{interest expense}}
\end{equation}
Wenn dieses Verhältnis im Verlauf über die letzten Jahre betrachtet wird, dann zeigt sich auch, ob das Unternehmen beispielsweise risikoreicher wird, wenn der Wert kleiner wird.\\

%
%%
\subsection{Current ratio}

Dieses Verhältnis beschreibt, wie viel kurzfirstiges Geld das Unternehmen nutzen kann, um alle kurzfristigen Schulden zu begleichen.
Die Werte hierzu finden sich im Balance Sheet.
\begin{equation}
    \formulaText{current ratio} = \frac{\formulaText{current assets}}{\formulaText{current liablities}}
\end{equation}
Gernerell ist davon auszugehen, dass sich bei einem Wert über $1.5$ keine kurzfristigen finanziellen Probleme beim Unternehmen abzeichnen, sollte plötzlich das kurzfristig geliehene Geld zurückgezahlt werden müssen.

Das \inventories wird auch zu den $\formulaText{current assets}$ gerechnet, kann jedoch ggf. nicht zu diesem Preis kurzfrisitg verkauft werden. 
Bei Unternehmen aus dem produzierenden Gewerbe, die einen signifikanten Anteil an Vermögenswerten als \inventories halten, sollte dies berücksichtigt werden und deshalb das sog. \textbf{Quick ratio} herangezogen werden.
\begin{equation}
    \formulaText{quick ratio} = \frac{\formulaText{current assets} - \formulaText{inventories}}{\formulaText{current liablities}}
\end{equation}
Hier deutet ein Wert um $1.0$ an, dass es keine kurzfristigen Probleme geben sollte, jedoch ist ein Vergleich mit den Konkurrenten aus der Branche hilfreicher als dieser Grenzwert.

%
%%
%%%
\section{Risiken}
Neben den genannten Kennzahlen sollten auch die Risiken des Unternehmens abgeschätzt werden.
Hierfür ist es hilfreich, aktiv nach Gründen zu suchen, die für einen Verkauf der Anteile am Unternehmen sprechen.
Dies können beispielsweise negative Analystenmeinungen und Prognosen sein oder auch Signale, die vielleicht in einigen Monaten für einen Verkauf sprechen würden, wie z.B. ein aktuell stattfindender Boom in der Branche, der aber nach einiger Zeit wieder nachlassen wird.

%
%%
%%%
\section{Management}
Das Management eines Unternehmens anhand von eindeutigen Kennzahlen zu beurteilen ist schwierig, da hier vielmehr die einzelnen Personen, wie z.B. der CEO, beurteilt werden sollten.
Auch ohne diese Personen persönlich kennenzulernen gibt es einige Anhaltspunkte, die einen Überblick über die Personen, die das Unternehmen führen, zu erlangen.

%
%%
\subsection{Vergütung (Compensation)}

Da Unternehmen diese Zahlen berichten müssen, lässt sich hierzu recht einfach ein Überblick erlangen.
Im sog. \glqq proxy statement\grqq\ ist unter Anderem aufgeführt, wie hoch die Vergütung der Vorstände ist.
Die \glqq summary compensation table\grqq\ listet auf, wie viel sich das Management für eine Periode auszahlen lässt.
Auch hier ist ein Vergleich zu anderen ähnlich großen Unternehmen aus der Branche hilfreich.

Die Vergütung des Managements setzt sich in der Regel aus einem fixen Anteil und einem performance-abhängigem Anteil, der von der Erreichung von Unternehmenszielen abhängig ist, zusammen.
Folgende Grundsätze sollten geprüft werden
\begin{itemize}
    \item Boni für eine Vergrößerung des Unternehmens, was durch den Kauf von anderen Unternehmen zustanden gekommen ist, sind als nicht gut zu bewerten.
        Diese stellen meist keine besondere Leistung des Managements dar.
    \item In guten Zeiten sollte den Managern mehr bezahlt werden, jedoch analog dazu in schlechten Zeiten auch weniger. Es sollte überprüft werden, ob dies der Fall ist.
    \item Halten die Vorstände Anteile am Unternehmen in Form von Aktien (und nicht nur Optionen)? 
        Falls diese kaum oder keine Aktien halten, dann sollte sich die Frage gestellt werden, ob diese Personen an einer positiven Entwicklung des Unternehmens interessiert sind.
\end{itemize}

%
%% 
\subsection{Charakter (Character)}

Die Beurteilung des Charakters von Vorständen und CEO ist ohne persönliche Kenntnis schwierig. 
Aber auch hier gibt es ein paar Möglchkeiten, sich einen Eindruck zu verschaffen.
\begin{itemize}
    \item Bereichern sich Angehörige der Personen im Management an dem Unternehmen, z.B. in Form von großen Aufträgen. 
        Im sog. \glqq 10-K-filling\grqq\ ist unter dem Punkt \glqq related-party transactions\grqq\ eine Übersicht, über diese Verbindungen.
    \item Die Biografie und die Historie der Personen im Aufsichtsrat liefert Auskunft, ob hier erfahrene Manager das Unternehmen leiten oder Neulinge mit wenig Erfahrung.
    \item Aus den Veröffentichungen des Unternehmens und der Medien zeigt sich, ob sich das Management auch Fehler eingesteht und diese auch an die Aktionäre kommuniziert.
        Im \glqq letter to shareholders\grqq\ lässt sich davon ein Eindruck gewinnen. 
    \item Es kann eingeschätzt werden, ob sich der CEO um das Unternehmen oder den Aktienkurs kümmert. 
        Wenn auch Entscheidungen getroffen werden, die sich zwar schlecht auf den Aktienkurs auswirken, aber das langfristige Überleben und den Erfolg des Unternehmens sichern, dann kümmert sich der CEO eher um das Unternehmen, was langfristig auch gut für die Aktionäre ist.
    \item Da die Personen in den Vorständen und im Aufsichtrat über der Zeit üblicherweise wechseln werden, kann hier Aufschluss über die allgemeine Stimmung im Management erhalten werden.
        Wenn die Personen oft wecheln, dann ist dies ein schlechtes Anzeichen, da es auch negative Auswirkungen auf die Stabilität eines Unternehmens haben kann.
\end{itemize}

%
%% 
\subsection{Operations}

Beurteilung der Performance des Unternehmens während der Amtszeit des Managements


\end{document}